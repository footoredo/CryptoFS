\documentclass{article}
\usepackage{enumitem}

\begin{document}

\title{\texttt{CryptoFS} Security Note}
\author{footoredo}
\date{\today}
\maketitle

\section{Basic structure}

\begin{itemize}[noitemsep]
    \item [] \texttt{\textbf{.cfs}}
    \begin{itemize}[noitemsep]
        \item [/] \texttt{keys}
        \begin{itemize}
            \item [/] \texttt{793220291197.key}
            \item [/] \texttt{dfc7027894e1.key}
            \item [] \texttt{......}
        \end{itemize}
        \item [/] \texttt{structure.sec}
        \item [/] \texttt{contents}
        \begin{itemize}
            \item [/] \texttt{1494eade37fa.key}
            \item [/] \texttt{9be1f33a5b86.key}
            \item [] \texttt{......}
        \end{itemize}
    \end{itemize}
\end{itemize}

\section{How to obtain the master key?}

\begin{itemize}
    \item [STEP 1] Retrive the unique identity of the device \texttt{\$DEVID}.
    \item [STEP 2] Ask for user passpharse \texttt{\$PASS}.
    \item [STEP 3] Compute \texttt{\$KEY = hashsum(\$DEVID + \$PASS)}.
    \item [STEP 4] Compute \texttt{\$ID = hashsum(\$KEY + \$PASS)}.
    \item [STEP 5] Find the key file \texttt{keys/\$ID[:12].key} and decrypt it using \texttt{\$KEY}.
\end{itemize}

\section{What's in the decrypted key file?}

\begin{itemize}
    \item [PART 1] Symmetric key \texttt{\$SIMKEY}
    \item [PART 2] Public-key encryption key-pair
    \begin{itemize}
        \item [KEY 1] Public key \texttt{\$PUBKEY}
        \item [KEY 2] Private key \texttt{\$PRIKEY}
    \end{itemize}
\end{itemize}

\section{\texttt{.sec} file}
A \texttt{.sec} file is the ecrypted version of the original file combined with digital signature to check its integerity. It can be decrypt into a corresponding \texttt{.raw} file.

\begin{itemize}
    \item [PART 1] Digital signature over hashsum of encrypted content (using \texttt{\$PUBKEY} and \texttt{\$PRIKEY}).
    \item [PART 2] Encrypted content (using \texttt{\$SIMKEY|\$SALT}).
\end{itemize}

\paragraph{Why use salt?} If not, two file with same content will have same encrypted content.

\section{\texttt{structure.sec}}

This file stores the directory structure of all original files. It is intended for implementation of \texttt{ls} command and operation validity check. Furthermore, it also stores the \texttt{\$SALT} for each file, which is needed in the section below, and the \texttt{stat} struct.

\paragraph{Why use salt?} Without salt, one can easily verify if a certain file exists. He just need to check if there is a file named \texttt{`hashsum("/foo/bar")[:12]`.sec} under the folder \texttt{contents}.

\section{Where to find a file?}

\begin{itemize}
    \item [STEP 1] Assume the dir for the file is \texttt{\$DIR}. First of all check if it is valid in \texttt{structure.sec}.
    \item [STEP 2] If it is valid, we can retrive \texttt{\$SALT} of this file. This file's identity can be computed in \texttt{\$ID = hashsum(\$DIR + \$SALT)}.
    \item [STEP 3] Find the corresponding \texttt{.sec} file \texttt{contents/\$ID[:12].sec} and decrypt it.
\end{itemize}

\section{Implementation with \texttt{fuse}}

\subsection{\texttt{mount}}


\begin{enumerate}
    \item 
At startup, first check if \texttt{.cfs} folder exists in the given mount point.

\begin{itemize}
    \item If found, query for the \texttt{\$PASS} and load keys.
    \item If not, query for the \texttt{\$PASS} as well and generate new keys.
\end{itemize}
    \item After keys are restored/generated, decrypt/create \texttt{structure.raw} file.
\end{enumerate}



\subsection{\texttt{umount}}

\begin{enumerate}
    \item Store \texttt{structure.raw} to \texttt{structure.sec}.
    \item Store keys.
\end{enumerate}

\subsection{\texttt{open}}

\begin{enumerate}
    \item Check if the file exists.
    \item Decrypt corresponding \texttt{.sec} file into \texttt{.raw} file.
\end{enumerate}

\subsection{\texttt{read}/\texttt{write}}

Just operate on the decrypted \texttt{.raw} file.

\subsection{\texttt{close}}

Encrypt \texttt{.raw} file into \texttt{.sec}.

\end{document}
